%_________________________________________________________________
%|                                  ihmtc.tex                    |
%| Template for IHMTC-2017 articles using LaTeX                  |
%| Created by   Dr. Supradeepan K                                |
%|              Department of Mechanical Engineering             |
%|              BITS-Pilani, Hyderabad Campus,                   |
%|              Hyderabad, India - 500 078.                      |
%|              Tel: (91) 40- (office)                           |
%|              Email: supradeepan@hyderabad.bits-pilani.ac.in   |                            
%|              WWW:                                             |
%|              May, 2017                                        |
%________________________________________________________________|


\documentclass[twocolumn,10pt,cleanfoot]{ihmtc}
\usepackage[T1]{fontenc}
\usepackage{balance}
\usepackage{amsmath}
\usepackage{amsfonts}
\usepackage{amssymb}
\usepackage{bm}


\usepackage{caption}
\usepackage{graphicx}
\graphicspath{{images/}}
\setlength{\belowdisplayskip}{5pt}
\setlength{\abovedisplayskip}{5pt}

\special{papersize=8.5in,11in}
\conffullname{the 24th National and 2nd International ISHMT-ASTFE\\
		Heat and Mass Transfer Conference (IHMTC-2017),}

%%%%% for date in a single month, use
%\confdate{24-28}
%\confmonth{September}
%%%%% for date across two months, use
\confdate{December 27-30}
\confyear{2017}
\confcity{BITS-Pilani, Hyderabad}
\confcountry{India}

%%% Use the number supplied to you 
%%% by Conference Secretariat for your paper
\papernum{IHMTC2017-06-0324}

%%% You need to remove 'DRAFT: ' in the title for the final submitted version.
%\title{\textsc{Paper Title}}
\title{Designed Periodic Cellular Materials for Enhanced Air-cooled Heat Sinks}
%%% first author
\author{Saksham Gakhar
\affiliation{Indian Institute of Technology Bombay\\saksham.gakhar94@gmail.com}
   
    }	

    

%%% second author
%%% remove the following entry for single author papers
%%% add more entries for additional authors
\author{Shankar Krishnan\thanks{Corresponding Author}
\affiliation{Professor, Indian Institute of Technology Bombay\\kshankar@iitb.ac.in}
   
    }	



\begin{document}

\maketitle 

\begin{abstract}
{\it
%---> Modify between these comments
Type in your abstract text here.
%---> Modify between these comments
}
\end{abstract}


\begin{nomenclature}
%---> Modify between these comments
\entry{}{\textbf{Symbols} (S.I.)}
\entry{$A$}{area of cross section}
\entry{$V,U$}{velocities}
\entry{$v,u$}{non-dimensional velocities}
\entry{$u_{mean-j}$}{mean pore velocity for case $ j $}
\entry{$d$}{diameter of pore}
\entry{$P$}{static pressure}
\entry{$n$}{number of (cylindrical) pores}
\entry{$L_C$}{length of conduit (laterally)}
\entry{$L_p$}{pore length}
\entry{$ A_s $}{cross-sectional area of porous screen}
\entry{$w$}{width of the conduit}
\entry{$C_{fi}$}{coefficient of loss due to fluid's turning from intake conduit into the pore}
\entry{$C_{fe}$}{coefficient of loss due to fluid's turning from the pore into the exhaust conduit}
\entry{$f_c$}{average skin friction coefficient in the cylindrical pore}
\entry{$Re$}{Reynolds number}
\entry{$ T $}{Temperature}
\entry{$q$}{heat transfer rate ($W$)}
\entry{$q'''$}{heat transfer rate per unit volume ($W/m^3)$}
\entry{$R_{thermal}$}{= $\Delta T / q'''$, thermal resistance}
\entry{$k$}{thermal conductivity}
\entry{$C_p$}{specific heat capacity}
\entry{$x,y$}{cartesian coordinates}
%%
\entry{}{\textbf{Greek symbols}}
\entry{$\beta$}{$=V/V_c$, ratio of the upstream velocity of fluid in the inlet conduit to the axial component of velocity of the fluid that branches off into (in case of $\beta_i$) or joins from (in case of $\beta_e$) the pore}
\entry{$\rho$}{mass density}
\entry{$\mu$}{dynamic viscosity}
\entry{$\tau$}{shear stress}
%%
\entry{}{\textbf{Subscripts}}
\entry{$i$}{for quantities in the inlet conduit}
\entry{$e$}{for quantities in the exit (or exhaust) conduit}
\entry{$c$}{lateral channel (or pore, cylinder)}
\entry{$w$}{wall}
\entry{$p$}{pore}
\entry{$ \infty $}{incoming ambient stream}
%---> Modify between these comments
\end{nomenclature}


\section{HEAT SINK TOPOLOGY}
%
Prospective `designed' periodic cellular materials like porous metal
foams not only offer lightweightedness and high effective surface
area (for a given volume with respect popular alternative heat fins)
but have also proved to be good acoustic absorbers. From a heat transfer
perspective, the large effective surface area is clearly a plus. We
are consiering the design of a heat sink where the aim is to draw
away heat from a heated base plate by blowing a fluid over the plate.
The process can be made efficient by deployment of fins which are
available in a variety of configurations, most popular ones being
the parallel plate and pin-shaped fins. But besides a good heat transfer
capability, another most essential requirement for an electronics
cooling application is low pressure drop across the sink. To this
end, we propose the use of a serpentine corrugated geometry of periodic
cellular material as the fin structure. Filling the entire heated
surface with metal foams looks like an attractive option from a heat
transfer vantage point but such a configurataion is bound to offer
unreasonably high pressure drops for the fluid medim required to be
blown over the sink, and thus does not quite qualify for an efficient
heat sink. Research has shown that although parallel plate fin configuration
offers extremely low pressure drops, the heat tansfer rate could still
be improved by the use of pin-fins but at the cost of a higher pressure
drop. The idea is to exploit the higher heat transfer capability of
periodic cellular porous materials and the low pressure drops offered
by the crrugated geometry such that the tradeoff between improved
heat transfer and lowered pressure drop typically encountered in other
popular alternatives is necessarily overcome.
%
\subsection{Periodic Unit Cell}
Fig. (\ref{}) shows a heat sink with periodic cellular material arranged in a serpentine manner over a heated plate. Considering applications in the electronics indutry, the cooling medium is typically air. The figure also shows the exploded view of a unit cell under consieration. An important approximation made at this juncture for the periodic porous fin structure is that it is composed of laterally stacked pores all with tortuosity equal to unity, thereby simplifying the pores to layer of laterally stacked cylinders. In practice, it is understood that this cellular structure is 3-dimensional and has a corresponding permeability matrix, we also approximate this porous screen by only a single layer pores that act as channels between the inlet conduit and the exit conduit. As a consequence, there is no permeability for flow between the cylinders or in other words, flow is restricted to take a 90$^\circ$ bend when it enters or leaves this cylindrical channel and thereby flow axially through these cylindrical pores. The top view of this approximate unit cell is shown in Fig. (\ref{}).
%%%%%%%%%%%%%%

\section{ANALYTICAL MODELLING}
\subsection{Flow Distribution}\label{BMmodel}
In order to develop a theoretical understanding for the flow distribution
across the model unit-cell and the pressure drop across it, the most
basic mass and momentum balance principles have been invoked on flow
elements in that intake and exit conduits. The resultant mathematical
model that describes the flow behaviour reveals an important non-dimensional
parameter ($m^2$) which broadly captures the effect of important
design parameters and whose relative magnitude (with respect to similar
alternate design configurations) necessarily reveals the degree of
the flow evenness. The model also provides insight on the scaling
of the pressure drop across the heat sink unit cell given a particular
value of $m^2$. \\
%
%%
%
All calculations for the model are based on one-dimensional flow equations
under simplifying assumptions. The panel of lateral pores is modelled
as of a single layer of adjacently stacked thin cylinders of diameter
$d$ and length $L_p$. These are assumed to have a wall thickness
$<<$ any of the characteristic dimensions. These lateral
pores form sharp junctions at right angles to either of the conduit
axes. In the dividing-flow- type inlet conduit, the fluid progresses
and as it approaches the lateral pore, some fluid is lost through
it and the remainder of the flow proceeds downstream at an ever decreasing
flow rate. In the combining flow case in the exit conduit, fluid approaches
the branch point along both the conduit and the lateral pore (cylinder).
These fluid streams combine at the branch point and proceed down-
stream along the manifold at an ever increasing flow rate. Primary
frictional losses are accounted for by $90^{\circ}$ turning losses,
pore wall friction losses and expansion-contraction losses (into and
out of the pores respectively). \\
%
%%
%
Consider an incompressible fluid flowing isothermally along the inlet
conduit of uniform cross- sectional area $A_i$ as illustrated by
Fig. (??). The control volume is shown with dotted line. A part of
the incoming flow branches into the lateral pore as a result of the
pressure difference in the inlet conduit and the ambient. $P_i(x)$
is the static pressure at a location x in the inlet conduit. Since
the flow decelerates due to loss of fluid to the lateral pore, the
pressure $P_i(x+\Delta x)$ is greater than the upstream pressure
$P_i(x)$. Fluid is transported to the lateral pore with a normal
velocity $U_c$ while retaining an axial velocity component $V_c$
as seen in Fig. (\ref{inletcon}). 
%
\begin{figure}[ht]
\centerline{\includegraphics[width=80mm,scale=0.50]{inletcon.PNG}}
\vspace{-1.5ex}
\caption{\small{A FLUID CONTROL VOLUME IN THE INLET CONDUIT. A FRACTION OF FLUID BRANCHES INTO THE PORE}}
\label{inletcon}
\end{figure}
%
Using the principle of mass balance, we thus
have : %
\begin{equation} \label{in-mass}
\rho {{A}_{i}}{{V}_{i}}=\rho {{A}_{i}}({{V}_{i}}+\frac{d{{V}_{i}}}{dx}.\Delta x)+\rho {{A}_{c}}{{U}_{c}}
\end{equation}
%
This when solved for the pore normal velocity gives
%
\begin{equation} \label{in-uc}
{{U}_{c}}\,=-\frac{{{A}_{i}}L_C}{{{A}_{c}}n}.\frac{d{{V}_{i}}}{dx} 
\end{equation}
%
The principle of momentum balance for the axial $x$ direction on the control volume yields
%
\begin{equation} \label{in-mom}
\begin{split}
{{P}_{i}}A_i+\rho {{A}_{i}}{{V}_{i}}^{2}={A}_{i}({{P}_{i}}+\frac{d{{P}_{i}}}{dx}\Delta x) & +{\tau }_{w}\pi {d}_{i}\Delta x \\
& +\rho {A}_{i}{({{V}_{i}}+\frac{d{{V}_{i}}}{dx}\Delta x)}^{2}+\rho {{A}_{c}}{{U}_{c}}{{V}_{c}}
\end{split}
\end{equation}
%
Now assuming the Darcy-Weisbach correlation ${{\tau }_{w}}=f_c\frac{\rho {{V}_{i}}^{2}}{8}$
to hold true in the conduit, and upon neglecting higher order terms
of $\Delta x$, we have :
%
\begin{equation} \label{in-mom-final}
\frac{1}{\rho }.\frac{d{{P}_{i}}}{dx}+\frac{{{f}_{i}}}{2{{d}_{i}}}{{V}_{i}}^{2}+(2-{{\beta }_{i}}){{V}_{i}}\frac{d{{V}_{i}}}{dx}=0
\end{equation} where $\beta_i=\frac{V_i}{V_c}$ . 
%
\begin{figure}[ht]
\centerline{\includegraphics[width=80mm,scale=0.50]{exitcon.PNG}}
\vspace{-1.5ex}
\caption{\small{A FLUID CONTROL VOLUME IN THE EXIT CONDUIT. A FRACTION OF INCOMING FLUID FROM THE PORE JOINS THE MAIN PASSAGE FLOW}}
\label{exitcon}
\end{figure}
%
Similar to the inlet conduit mass balance eqs. (\ref{in-mass},\ref{in-uc}), the mass balance on control volume shown in Fig. (\ref{exitcon}) on the exit conduit gives
%
\begin{equation} \label{out-uc}
{{U}_{c}}\,=\frac{{{A}_{e}}L_C}{{{A}_{c}}n}.\frac{d{{V}_{e}}}{dx}
\end{equation}
%
Momentum balance principle on the control volume shown in Fig. (\ref{exitcon}) on the exit conduit yields (similar to eq. (\ref{in-mom-final}))
%
\begin{equation} \label{out-mom-final}
\frac{1}{\rho }\frac{d{{P}_{e}}}{dx}+\frac{{{f}_{e}}}{2{{d}_{e}}}{{V}_{e}}^{2}+(2-{{\beta }_{e}}){{V}_{e}}\frac{d{{V}_{e}}}{dx}=0
\end{equation}
%
Eqs. (\ref{in-uc}) and (\ref{out-uc}) together yield the following
relation between the inlet and the exit conduit velocities at any
$x$under the boundary condition $V_e=0$ at $x=0$ and $V_i=V_o$ at $x=0$ :
%
\begin{equation} \label{vel-relation}
{{V}_{e}}=\left( {{V}_{o}}-{{V}_{i}} \right)\frac{{{A}_{i}}}{{{A}_{e}}}
\end{equation}
%
Now subtracting eq. (\ref{out-mom-final}) from eq. (\ref{in-mom-final}),
we get the following equation : 
%
\begin{equation} \label{combined-mom}
\begin{split}
\frac{1}{\rho}\frac{d({{P}_{i}}-{{P}_{e}})}{dx}+\frac{{f}_{i}}{2{{d}_{i}}}{{V}_{i}}^{2}-\frac{{f}_{e}}{2{{d}_{e}}} & {{V}_{e}}^{2} +(2-{\beta}_{i}){V}_{i}\frac{d{{V}_{i}}}{dx}\\ 
& - (2-{\beta }_{e}){V}_{e}\frac{d{{V}_{e}}}{dx}=0
\end{split}
\end{equation}
%
Additionally, we write the pressure drop for a pore at a location
$x$ as follows : 
%
\begin{equation} \label{zeta-def}
{{P}_{i}}-{{P}_{e}}=\rho(1+{{C}_{fi}}+{{C}_{fe}}+{{f}_{c}}\frac{{{l}_{c}}}{{{d}_{c}}}){{U}_{c}}^{2}=\rho \zeta \frac{{{U}_{c}}^{2}}{2}
\end{equation}where $C_{fi}$ is coefficient of turning loss from the intake conduit
into the lateral pore and $C_{fe}$ that of turning loss from the
lateral pore into the exhaust conduit, and $f_c$ is average friction
coefficient for the flow through the pore.
%
Now using relations in eqs. (\ref{in-uc},\ref{out-uc},\ref{vel-relation},\ref{zeta-def})
to simplify eq. (\ref{combined-mom}) above and also non-dimensionalising
the equation with ${{p}_{i}}=\frac{{{P}_{i}}}{\rho{{V}_{o}}^{2}},{{p}_{e}}=\frac{{{P}_{e}}}{\rho{{V}_{o}}^{2}}{{v}_{i}}=\frac{{{V}_{i}}}{{{V}_{o}}},{{v}_{e}}=\frac{{{V}_{e}}}{{{V}_{o}}},{{u}_{c}}=\frac{{{U}_{c}}}{{{V}_{o}}},X=\frac{x}{L}$,
we get the following control equation that governs the velocity variation : 
%
\begin{equation} \label{controleq}
\frac{{{d}^{2}}(1-{{v}_{i}})}{d{{X}^{2}}}-{{m}^{2}}(1-{{v}_{i}})=\varepsilon
\end{equation}%
%
where 
%
\begin{equation} \label{msq}
\varepsilon\,=\frac{\left( 2-{{\beta }_{i}} \right)}{\zeta{{\left[\frac{{{A}_{i}}}{n{{A}_{c}}} \right]}^{2}}}\,\,\,\,\text{   and    }\,\,\,\,{{m}^{2}}=\left\{ \left[ \frac{2-{{\beta }_{e}}}{2-{{\beta }_{i}}} \right].{{\left[ \frac{{{A}_{i}}}{{{A}_{e}}} \right]}^{2}}-1 \right\}.\varepsilon
\end{equation}
%
%
This control equation can be solved under the known boundary conditions
$v_i=1$ at $x=0$ and $v_i=0$ at $x=1$. The solution will be different
depending on the sign of $m^2$. It has been found from a large set
of computational data that for all practical purposes $\beta_i\simeq0.83$
and $0.02 \leq \beta_e \leq 0.1$. For such a case, the value of $m^2$
is necessarily $>0$. Although eq.(\ref{controleq}) could also be solved for cases where $m^2\leq0$, we present the solution for cases
where $m^2>0$.\\
%%%%%%%%%%%%%%%%%m_sq>0 solution%%%%%%%%%%%%%%%%%%%%%%%%%%
\textbf{Inlet Conduit velocity ($\bm{v_i}$)}
\begin{equation}\label{vi} 	
\begin{split}
{v}_{i}=1-\left( \frac{\sinh (mX)}{\sinh (m)} \right)+\frac{\varepsilon }{{m}^{2}}\Biggl[ 1 & - \left( \frac{\sinh (mX)}{\sinh (m)} \right) \\
& -\left( \frac{\sinh (m(1-X))}{\sinh (m)} \right) \Biggr]
\end{split}
\end{equation}
%
%
\textbf{Exit Conduit velocity ($\bm{v_e}$)}
\begin{equation}\label{ve}
{v}_{e}=\left( 1-{v}_{i} \right)\frac{{A}_{i}}{{A}_{e}}
\end{equation}
%
%
\textbf{Channel Velocity ($\bm{u_c}$)}
\begin{equation} \label{uc} 
\begin{split}
{u}_{c}=\left(\frac{{A}_{i}}{{{A}_{c}}n} \right) & \left( \frac{m}{\sinh (m)} \right) \Biggl[ \cosh (mX)  \\ 
 & + \frac{\varepsilon }{{m}^{2}}\biggl( \cosh (mX)-\cosh (m(1-X)) \biggr) \Biggr]
\end{split}
\end{equation}
%
%
\textbf{Local pressure drop ($\bm{p_i-p_e}$)}
\begin{equation} \label{dp} 
\begin{split}
{p}_{i}-{p}_{e}=\frac{\zeta}{2} & \left( \frac{{A}_{i}}{{{A}_{c}}n} \right)^{-2}\; {\biggl( \frac{m}{\sinh (m)} \biggr)}^{2}\Biggl\{ \cosh (mX)\\
& +\frac{\varepsilon }{{m}^{2}}\biggl( \cosh (mX) - \cosh (m(1-X)) \biggr) \Biggr\}^{2}
\end{split}
\end{equation}
%
%
\textbf{Net pressure drop ($\bm{\Delta P}$)}
\begin{equation} \label{netdp} 
\begin{split}
\Delta P=\biggl({{\left. {{p}_{i}} \right|}_{X=0}}-{{\left. {{p}_{e}} \right|}_{X=1}}&\biggr)=\frac{\zeta}{2} {\left( \frac{{A}_{i}}{{{A}_{c}}n} \right)}^{-2}\; \Biggl\{{{\left( \frac{m}{\tanh (m)} \right)}^{2}} \\
& +\varepsilon \left( 1+\frac{\varepsilon }{{{m}^{2}}} \right){{\left( \frac{\left( 1-\operatorname{sech}(m \right)}{\tanh (m)} \right)}^{2}} \Biggr\}  
\end{split}
\end{equation} 
%
%
\subsection{Existence of Optimal Geometry}\label{existence_dopt}
The characteristic length scales for heat transfer devices including but not limited to heat sinks pose the fundamental problem of demarcation of permissible optimal ranges of these lengths from a design perspective; the aim of which is to maximise heat transfer density through the device within a fixed volume. In our case, considering the flow through laterally stacked cylinder as shown in Fig. \ref{}, an essential component of the design cycle is the choice of  optimal flow channel size and optimal porosity of the assembly. The problem becomes an unconventional one because with the increasing miniaturisation of scales, the regimes of operation tend to those where the slenderness assumption starts to break ($d/L_C$ not $<< 1$) along with which common heat transfer premises such as boundary layer theories and correlations also start proving to be invalid and misleading. This problem of showing the existence of and henceforth choosing an optimal design range for channel (pore) size has been addressed in the section. The latter, i.e. the choice of the optimal sizes, is also governed in part by other macroscopic requirements for overall heat transfer density like increased degree of flow uniformity as addressed in section \ref{choiceofdesign}.
\subsubsection*{Optimal geomerty in the small-scale limit:} In the limit of decreasing length scales, consider the limit of dimensions $ d $ and $ L_C $ such that $d/L_C$ is not $<< 1$. There are three flow configurations under which heat sinks are typically used depending on how the heat sink is attached to the cooling network \cite{14th from wen lu}. These are : first, a fixed pumping power; second, a fixed pressure drop; and third, a fixed mass flow rate. Here we consider the case where the pressure drop $ \Delta P $ across the heat sink and hence the periodic unit cell under study is fixed. The analysis presented next can nevertheless be extended to the other two configurations as well by appropriately changing the relation between the mean velocity and $\Delta P$.\\
%
In case $ (a) $, the first extreme, there are so many pores packed in the same volume such that the diameter $ d $ is so tight that owing to excellent wall-fluid thermal contact, the exiting low through the pore can be assumed to be at the same temperature as the wall ($ T_w $). In other words, the flow can be seen to be approximately like a Hagen-Poiseuille \emph{`type'} fully developed channel flow. In this case $ (a) $ therefore, we can approximate the mean pore axial velocity to be 
%
\[u_{mean-a}=\frac{{d}^{2}}{32\mu }\frac{\Delta P}{{L}_{p}}\]
%
Since the cylinders have very thin walls (with thicknesses $\sim d/100 $) the total flow area for the porous screen, $ A_s $, is approximately equal to $n \times$ {\it cross sectional area of one pore}. Therefore, the total heat transfer is $ q=\dot{m}C_p(T_w-T_\infty) $, where $ T_\infty $ is the temperature of the incoming stream of cooling fluid. Here $\dot{m}=\rho A_s u_{mean-a}$. Therefore we have, the heat transfer density as $ q''' =q/V$ where $ V $ is the volume to which heat is rejected ($ =A_s \times L_p$). Thus,
%
\begin{equation}\label{d0}
q'''({W}/{{m}^{3}})=\frac{\rho {{C}_{p}}(T_w-T_\infty)\Delta P}{32\mu }\frac{{{d}^{2}}}{{{L}_{p}}^{2}}
\end{equation}
%
In the other extreme case $ (b) $ of $d/L_C$ is not $<< 1$, we have the diameters that are large enough with respect to pore lengths such that the channels are no longer slender. In this increasing $d/L_C$ scenario, the boundary layer slenderness assumption breaks and contribution of convection mode of heat transfer becomes negligible with respect to conduction. Let's assume a small thickness of fluid film that surrounds the wall inside the channel to be $ \delta $ which can be approximated, through dimensional analysis as $ 2\sqrt{\alpha t} $, where $t$ has the dimensions of time. It follows that,
\[\delta \simeq 2\sqrt{\alpha \frac{L_p}{u_{mean-b}}}\]
Furthermore, making the Bernoulli approximation, we have,
\[u_{mean-b} \simeq \sqrt{\frac{2\Delta P}{\rho }}\]
Now, the conduction heat transfer from the wall to this fluid film can be written as $ q_1=k S (T_w-T_\infty)$, where $ S $ is the thermal shape factor for the film developed close to the wall. This is the heat transfer rate through one channel. Under the assumption of a near cylindrical film of thickness $\delta$ (with the inner radius of the annular cylinder being $d-\delta$ and outer being $d$), the expression becomes
\[q_1=k \left( \frac{2 \pi L_p}{ln\left(\frac{d}{d-\delta}\right)} \right) (T_w-T_\infty) \]
Now we can write : $ln(\frac{d}{d-\delta})=ln(1+\frac{\delta}{d-\delta}) \Rightarrow ln(1+\frac{\delta}{d-\delta}) \simeq \frac{\delta}{d-\delta}$. (This simplification uses the approximation : $ln(1+x) \simeq x$ if $x<<1$, where $x=\frac{\delta}{d-\delta}<<1$ in our case). Futhermore, we have $ln(\frac{d}{d-\delta})\simeq\frac{\delta}{d}$ because $d-\delta \simeq d$. The number of pores again $n=A_s/(\frac{\pi d^2}{4})$. Therefore total heat transfer rate through $n$ pores becomes
\[q=\biggl( \frac{4 A_s}{\pi d^2}\biggr) \frac{2 \pi k L_p d}{\delta}(T_w-T_\infty)\]
We are interested in the heat transfer density $q''' = q/V$, where again $V=A_s \times L_p$. Finally we get, 
%
\begin{equation}\label{dinf}
q'''(W/m^3)=\frac{4k (T_w-T_\infty)}{d}\left(\frac{2\Delta P}{\rho\alpha^{2}L_{p}^{2}}\right)^{1/4}
\end{equation}
%
Eqs. (\ref{d0}) and (\ref{dinf}) are the two asymptotes and these two clearly prove the existence of an optimal diameter of the pore for maximal heat transfer density (since eq. \ref{d0} has $q''' \propto d^2$ and eq. \ref{dinf} has $q''' \propto d^{-1}$). Therefore, by the method of matched asymptotes \cite{bejan2004declength}, $q'''$ is warranted to peak in the vicinity of the intersection of Eqs. (\ref{d0}) and (\ref{dinf}). Therefore the optimal spacing and maximal heat transfer density is given as 
%
\begin{equation}
\left(\frac{d}{L_{p}}\right)_{optimal} \simeq \frac{\left(128.\mu\right)^{{1}/{3}}}{\left(\Delta P \, L_p^2 \right)^{{1}/{4}}}.\left(\frac{2\alpha^{2}}{\rho}\right)^{{1}/{12}}
\end{equation}
%
\begin{equation}
q_{max}\,  \lessapprox \, \frac{k (T_w-T_\infty)}{L_p}\frac{\sqrt{\Delta P}}{(2\, \rho \mu^2 \alpha^4)^{1/6}}
\end{equation}









%%%%%%%%%%%%%%%%%%%%%%%%%%%%%%%%%%%%%%%%%%%%%%%%%%%%%%%%%%%%%%%%%%%%%
\section{COMPUTATIONAL MODELLING}\label{compmodel}
To examine the validity and accuracy of the solutions obtained using the above analytical model, the flow distribution and heat transfer is studied computationally where the self-repeating unit cell shown in Fig. (\ref{}) has been chosen as the calculation domain for simulation in ANSYS Fluent commercial software. The most general system of governing equations for the single-component fluid which describe the mean flow properties, is cast in integral Cartesian form for an arbitrary control volume $V$ with differential surface area $d\boldsymbol{A}$ as follows
%
\begin{equation*}
\frac{\partial}{\partial t}\int_{V}\boldsymbol{W\:}dV+\iint_{A}\,[\mathbf{F}-\mathbf{G}].d\boldsymbol{A}=\int_{V}\boldsymbol{H\:}dV 
\end{equation*}
%
where the vectors $\boldsymbol{W}$, $\mathbf{F}$ and $\mathbf{G}$ are defined as 
%
\begin{equation*}
 \boldsymbol{W}=\left[\begin{array}{c} \rho\\ \rho u\\ \rho v\\ \rho w\\ \rho E \end{array}\right] ; \boldsymbol{\mathbf{F}}=\left[\begin{array}{c} \rho\boldsymbol{v}\\ \rho\boldsymbol{v}u+\rho\boldsymbol{i}\\ \rho\boldsymbol{v}v+\rho\boldsymbol{j}\\ \rho\boldsymbol{v}w+\rho\boldsymbol{k}\\ \rho\boldsymbol{v}E+p\boldsymbol{v} \end{array}\right] ; \mathbf{G}=\left[\begin{array}{c} 0\\ \tau_{xi}\\ \tau_{yi}\\ \tau_{zi}\\ \tau_{ij}v_{j}+\boldsymbol{q} \end{array}\right]
\end{equation*}
%
and $\boldsymbol{H}$ vector contains source terms such as gravity body-force and the heat generation terms. Here $\rho$, ${\mbox{\boldmath$v$}}$, $E$ and $p$ are the fluid density, velocity, total fluid energy per unit mass and pressure respectively. ${\mbox{\boldmath$\tau$}}$ is the viscous stress tensor, and ${\mbox{\boldmath$q$}}$ is the heat flux. \\
%
A second-order upwind flux scheme was used for the flow and the SIMPLE finite volume algorithm was used for obtaining the velocity fields. The calculations were terminated when the scaled residuals had dropped below ${{10}^{-7}}$ for all governing equations. The system is solved for steady state, incompressible, laminar and viscous flow of air ($\mu =1.81\times {{10}^{-5}}~Pa$ and $\rho =1.225\,\text{kg/}{{\text{m}}^{3}}$). The thermo-physical properties have been assumed constant. 
%
The boundary conditions used in the simulations are enumerated next.
\begin{enumerate}     
\item {A velocity inlet boundary condition with a uniform value was assumed at the entrance of the inlet conduit. The inlet pressure was set to 50 $Pascals$ of gauge pressure. The inlet velocity magnitude for all the simulations is 1 $m/s$. The temperature of the incoming cooling fluid (air) has been assumed to be equal to the ambient (300 K).}
%         
\item{A pressure outlet boundary condition was imposed on the outlet of the domain with nil gauge pressure. The outlet is also assumed to be at ambient.}
%         
\item{The top walls of both the inlet and exit conduits (numbered ** in Fig. \ref{}) and the rear walls of each conduit (numbered ** in Fig. \ref{}) are assumed to be stationary with the no-slip boundary condition. The constant temperature at these walls is assumed to be the same as ambient (300 K). }
%
\item{The bottom walls (numbered ** in Fig. \ref{}) and the walls of the laterally stacked cylinders are also assumed to be stationary with no-slip boundary condition. Here, unlike the top walls, a constant temperature thermal boundary condition has been imposed. Basically, thermal resistance in the cell walls has been ignored ($k_w \rightarrow \infty$). The bottom plate and the cylindrical fins are assumed isothermal at 343 K.}
%       
\item{The two walls that lie along the length of the conduit (numbered ** in Fig. \ref{}) form the symmetry planes for the periodic unit cell. We impose the symmetry boundary condition to these walls which is equivalent to a 0 shear stress along the wall hydrodynamically. \cite{fluent} This condition is invoked owing to the symmetry of the simulaton geometry.}      
\end{enumerate}
%
%%%%%%%%%%%%%%%%%%%%%%%%%%%%%%%%%%%%%%%%%%%%%%%%%%%%%

\section{RESULTS AND DISCUSSION}
%
\subsection{Validation of the model}
Although it is feasible to reliably model and predict the fluid velocities and pressure drops in a prospective heat sink through numerical simulations, the path to such an analysis may not be the best when a suitable model, such as the one discussed in section \ref{BMmodel} can be used instead. In this section we present the comparison of local pressure drops across each of the cylindrical pores and inlet \& exit conduit velocity variations. The results have been obtained from the analytical model of section \ref{BMmodel} and from numerical simulations with specifics discussed in section \ref{compmodel}.
\subsubsection*{Inlet and exit conduit velocities:} The 


\subsection{Validation for Existence of Optimal Geometry}
We theoretically postulated the existence of an optimal channel size ratio ($d/L_p$) and a maximal heat transfer density $q'''_{max}$ in section \ref{existence_dopt}. In this section, we numerically prove the existence of such an optimal configuration. It is not easy to obtain the optimal channel size numerically for any given configuration since it is an expensive task both computation-wise and in time to generate unit cells for different configurations. However, it is relatively easier to validate the theoretical prediction for the optimal sizing by varying the diameter $d$ (holding length of pore, $L_p$ constant) and studying the variation of the heat transfer for different conduit lengths ($L_C$, holding width $w$ constant). Variation in $d$ for a given $L_C$ is effectively same as varying the porosity (or number of pores). Looking at the maximal $q'''$ is equivalent to looking at the minimum thermal resistance $R_{thermal}$ $(Kelvin-m^3/W)$ where $R_{thermal}=(T_w-T_{\infty})/q'''$. For a fixed pressure drop of 50 $Pa$, inlet velocity of 1 $m/s$, $w=0.025m$, $L_p=0.01m$, Figs. (\ref{RminLC/w_a}) and (\ref{Rmin_b}) show the existence of minimum $R_{thermal}$. The dotted curves in these figures are for the two asymptotes of eqs. (\ref{d0}) \& (\ref{dinf}). It is clear that with decreasing values of $L_C/w$, the minima shifts towards smaller and samller values of $d/L_p$. It is because of this reason that the minima is not discerbible in Fig. (\ref{RminLC/w_c}). Nevertheless, all these CFD predictions for the optimal configuration that correponds to minimum $R_{thermal}$ are well within the vicinity of the optimal predicted through asympotic analysis, thus demonstrating the feasibility of the method.  Although the model has somewhat underpredicted the value of $R_{thermal}$, the qualitative variations are very close to what one would expect.

\subsection{Optimal Channel Topology}\label{choiceofdesign}
\subsubsection*{Flow Distribution \& Need for Uniformity:} The heat transfer has to be optimised on two fronts. The macroscopic scale optimisation commands that the flow distribute evenly along the length of the inlet conduit; which is to say that the flow must, in the ideal case, divide itself uniformly as it leaks out of the inlet conduit into the exit conduit via the laterally stacked pores. The system, if not designed for the optimal case, will see the fluid entering the inlet conduit and taking the least resistance path which is the open conduit itself. In such a case, maximum fluid transitions to the exit conduit passing only through a few pores located at the opposite end of the inlet conduit, while very little fluid bends ito the pores on its way downstream in the inlet conduit. As a result, most of the length of porous material located close to the inlet will remain untouched by the fluid. This in turn implies that the system's surface area exposed for heat transfer by forced convection is effectively very small and thereby the system does not perform maximally. And hence the need for ensuring good uniformity of flow in the lengthwise direction through a suitable choice of geometric design parameters.
\subsubsection*{Choice of design:}


%%%%%%%%%%%%%%%%%%%%%%%%%%%%%%%%%%%%%%%%%%%%%%%%
\subsection{Optimal Fins per Inch}\label{fpi}
So far we have considered a periodically repeating unit cell. But what are the design rules for determination of the number of fins (or unit cells) that can be fitted in a given length of the heated plate over which the heat sink fins have to be mounted? This can be addressed by simulatneously looking at the behaviour of $R_{thermal}$ and deviations from flow uniformity as a function of variable width of the conduits, $w$.



\bibliographystyle{ihmtc}
\bibliography{ihmtc}


\end{document}
